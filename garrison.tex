%!TEX TS-program = xelatex
\documentclass{nihbiosketch}

% \usepackage{draftwatermark}  % delete this in your document!
% \SetWatermarkText{Erik Garrison UCSC}    % delete this in your document!
% \SetWatermarkLightness{0.9}  % delete this in your document!

\usepackage{mathpazo} % add possibly `sc` and `osf` options
\usepackage{eulervm}
\usepackage{comment}
\usepackage{pifont}
\hypersetup{nolinks=true} % disable hyperlinks in citations

\usepackage[utf8]{inputenc} % Required for inputting international characters
\usepackage[T1]{fontenc} % Output font encoding for international characters
% A note on fonts: As of 2019, NIH allows Arial, Georgia, Helvetica, and Palatino Linotype. Georgia and Arial are commercial fonts so you will need to use XeLaTeX and have them installed on your machine to use them. Palatino & Helvetica are available as free LaTeX packages so select the one you want and comment out the other.
\usepackage{palatino} % Palatino font
\linespread{1.05} % A little extra line spread is better for the Palatino font
%\usepackage{helvet} % Helvetica font
\renewcommand*\familydefault{\sfdefault} % Use the sans serif version of the font


%\usepackage{palatino} % Palatino font
%\linespread{1.05} % A little extra line spread is better for the Palatino font

\usepackage[dvipsnames]{xcolor}

\pagestyle{myheadings}
\pagenumbering{gobble}
%\input{../version}
% \markright{{\color{gray}\tiny \hfill\VERSION~(\today)~~}}
\newcommand{\journal}[1]{{\bf #1}}
\newcommand{\hijournal}[1]{{\bf {\uline{#1}}}}

% \renewcommand{\labelitemiii}{$\diamond$}

\newcommand{\course}{
  \emph{Advanced bioinformatics: data
    mining~\&~data integration for Life Sciences}}

%------------------------------------------------------------------------------

\name{\uline{Garrison}, Erik}
\eracommons{EGARRISON}
\position{Assistant Professor}

\begin{document}
%------------------------------------------------------------------------------

\begin{education}
  Harvard University & A.B. & 05/2006 & Social Science \\
  Cambridge University & Ph.D. & 01/2019 & Genomics \\
\end{education}

\section{Personal Statement}

\begin{statement}

  %I build computational systems to assemble genomes and relate populations of genomes.
My research explores genome structure, diversity, and evolution using tools that establish sequence-level relationships between many genomes.
These \emph{pangenomic methods} provide a unified, graphical approach to comparative and population genomics.
They also support common basic bioinformatics applications like resequencing and variant calling, where they improve our resolution of variable regions of the genome and reduce our bias to all kinds of variation.
My work anticipates that in the near future, genome assembly will be a solved problem, and instead we will face a new challenge of understanding and interpreting large collections genomes and their relationships.
%I began my scientific journey as a research assistant in sociology and economics.
%To enable a study of policy effects on Wikipedia article creation, I built a streaming data processing system capable of quickly processing the tens of terabytes of text data in the history of the website.
Generally, I have worked on the topic of \emph{genome inference}.
My work on this topic began with the development of Bayesian methods to detect and genotype genomic variants, with application of these methods to the thousands of human genomes cataloged in the 1000 Genomes Project.
Lessons learned in that effort guided me to work on unbiased methods for genome inference based on graphical models of pangenomes.
In these, the genome is encoded in a graph that may represent a population sample of individuals from the same species, a metagenome, the diploid genome of a single individual, or any other useful collection of genomic sequence information, including those made from genomes of different species.
Working as part of a world-spanning collaboration in computational pangenomics, I have helped to demonstrate the utility of this approach for basic bioinformatics analyses like read alignment and genotyping.
The addition of information about population variation to the reference system improves our resolution of genetic variants of all scales.
As the cost of assembling genomes dropped, combinations of multiple sequencing technologies enabled the semi-automated generation of reference-quality genomes for many species.
Although I have not worked on assembly algorithms directly, I have supported genome assembly pipelines like that used by the Vertebrate Genomes Project, where I helped to apply haplotype-based variant detection methods for assembly polishing.
These assemblies are the primary material of my current work, which focuses on the assembly of collections of genomes into \emph{pangenome graphs}.
I am part of the leadership of the ``Pangenomes Working Group'' of the Human Pangenome Reference Consortium (HPRC).
There, I have focused on building pangenome models from all-to-all alignment of genome sequences, in a manner that does not depend on any reference sequence or input order.
When applied to the high-quality assemblies of the HPRC, this unbiased approach allows us to understand features of human genome variation that have never before been observed so directly.
%As an independent PI, I have focused strongly on the problem of pangenome graph construction, resulting in the first

%We find that we can improve our resolution of variants of all scales, both small (as in SNPs and indels) and large (structural variants).


%I am currently working to improve the efficiency of related approaches, and exploring the relationship between the genome assembly and pangenome construction problems.
%I am part of a project to create an Australian aboriginal pangenome, with the objective of improving clinical inference for these underrepresented groups.
%I am funded to develop privacy preserving variation graphs, which will support the sharing of genomic information from private sources without risking patient privacy.
%I will be part of the leadership of the pangenomics working group in the newly-formed Human Pangenome Reference Consortium, where I will scale my approaches to help build a pangenome from hundreds of high quality human genome assemblies.


% 4 general references related to application

\begin{enumerate}[label=\alph*.]

\item \textsc{Erik Garrison}, Jouni Sirén, Adam M Novak, Glenn Hickey, Jordan M Eizenga, Eric T Dawson, William Jones, Shilpa Garg, Charles Markello, Michael F Lin, et al. \emph{Variation graph toolkit improves read mapping by representing genetic variation in the reference}. \hijournal{Nature Biotechnology}, 36(9):875–879, 2018. PMCID: PMC6126949

%\item 1000 Genomes Project Consortium et al. \emph{An integrated map of genetic variation from 1,092 human genomes}. \hijournal{Nature}, 491(7422):56, 2012. PMCID: PMC3498066

\item 1000 Genomes Project Consortium et al. \emph{A global reference for human genetic variation}. \hijournal{Nature}, 526(7571):68, 2015. PMCID: PMC4750478

\item Peter Sudmant, Tobias Rausch, Eugene J Gardner, Bob Handsaker et al. \emph{An integrated map of structural variation in 2,504 human genomes}. \hijournal{Nature}, 526(7571):75, 2015. PMCID: PMC4617611

\item Arang Rhie, Shane A McCarthy, Olivier Fedrigo, Joana Damas et al. \emph{Towards complete and error-free genome assemblies of all vertebrate species}. \hijournal{Nature}, 592(7856):737, 2021.

\end{enumerate}

\end{statement}

%------------------------------------------------------------------------------
\section{Positions and Honors}

\subsection*{Professional Experience}
\begin{datetbl}
2020--now  & Assistant Professor, University of Tennessee Health Science Center \\
2019--2020 & Postdoctoral fellow with Benedict Paten, University of California, Santa Cruz, USA \\
2014--2018 & PhD student with Professor Richard Durbin, Wellcome Sanger Institute, UK \\
2015--2018 & Contractor, DNAnexus Inc., San Francisco, CA, USA \\
2010--2014 & Research scientist with Gabor Marth, Boston College, USA \\
2009-- & Contractor, The Echonest, Somerville, MA, USA \\
2008--2009 & Software Engineer, One Laptop Per Child, Cambridge, MA, USA \\
2006--2008 & Contractor, George Church lab, Harvard Medical School, Boston, MA, USA \\
2006--2007 & Research assistant, National Bureau of Economic Research, Cambridge, MA, USA \\
2005-- & Research assistant, Harvard Kennedy School of Government, Cambridge, MA, USA \\
\end{datetbl}

%\pagebreak

%------------------------------------------------------------------------------

\section{Contribution to Science}

\begin{enumerate}

\item Haplotype-based genetic variant detection

In resequencing, we align reads from one or many new genomes to a reference genome, then use the joint information available in this data to infer the genotype of the resequenced genomes at every position in the reference. A core problem in this process is variant calling, in which candidate alleles are discovered in the collection of reads and genotypes are estimated for each sample under consideration. Early work on this topic focused on the detection of single base changes from the reference using disaggregated base observations. This approach proved error-prone and is fundamentally unable to accurately detect indels and complex small variation due to the problem of overlapping variants. In freebayes, I implemented a haplotype-based approach that provides a coherent local description of variation in the reads at a given locus, irrespective of allele type. It produces variant calls over local haplotypes, rather than point-wise alleles. Since its introduction, this method has seen increasing use in the biological community. Due to the configurability and minimal assumptions made by the method, it is particularly useful in unusual genomic contexts that are underserved by methods adapted to the low diversity setting found in humans. It has been applied to understand the genesis of a Cholera outbreak in Haiti, and to an exobiological study of bacterial isolates from the International Space Station. It is used to call genotypes in clinical genomics, run realtime in a rapid DNA sequencing device, and applied to the genome assembly polishing problem. Through my support and training of users of this method, I have learned about the whole breadth of biology. My work on this topic demonstrated the fundamental problem of resequencing against a linear reference model, and motivated my work on variation graphs. Freebayes has supported thousands of studies (73,000 hits on Google as of Nov 11, 2019), and a preprint documenting its basic Bayesian model has been cited 1204 times (Erik Garrison and Gabor Marth. Haplotype-based variant detection from short-read sequencing. arXiv:1207.3907, 2012, \url{https://arxiv.org/abs/1207.3907}). I additionally supported the development of an optimized resequencing pipeline based on freebayes:

  \begin{enumerate} % bio1

\item Colby Chiang, Ryan M Layer, Gregory G Faust, Michael R Lindberg, David B Rose, \textsc{Erik P Garrison}, Gabor T Marth, Aaron R Quinlan, and Ira M Hall. \emph{SpeedSeq: ultra-fast personal genome analysis and interpretation}. \hijournal{Nature Methods}, 12(10):966, 2015. PMCID: PMC4589466

  \end{enumerate}


\item Integration of a human pangenome in the 1000 Genomes Project and Human Pangenome Reference Consortium

  The 1000 Genomes Project explored the scope of human genome variation by resequencing approximately 2500 human genomes from 25 diverse world populations. This work engaged a large community of researchers who were developing methods to work with the low-cost, short read sequencing data around which the project was based. I worked with the project analysis group to develop a strategy for the integration of results from dozens of variant calling methods and related data sources, resulting in a phased collection of genomes that has since served the role of a human pangenome reference to the research community. I applied the variant calling techniques I developed in this process, and explored graph based alignment methods for the resolution of complex variation in the final project release, which I produced under the guidance of the analysis group and in collaboration with researchers developing novel imputation and phasing methods for complex variation. I have guided the application of the integration approach that we developed to other similar problems, including the determination of germline variation in the PanCancer Analysis of Whole Genomes project. This experience taught me a great deal about the complexity of establishing high-quality data resources, and it made me intimately aware of the pitfalls of resequencing against a linear reference genome, further motivating my work on pangenomic reference systems. While participating in this project, I supported numerous related studies that examined selection, demography, structural variation, transposable elements, and supported the development of high-performance alignment and phasing techniques capable of efficiently operating on thousands of human genomes. In later work, I have guided others in the application of similar techniques to the study of African genomes (in the Uganda genome project) and cancer genomics (Pan-Cancer analysis of whole genomes). Presently, I am a participant and working group chair in the Human Pangenome Reference Consortium. We have recently provided our first result: a pangenome built from 47 haplotype-resolved human genomes. I supported this work as corresponding author, and my group contributed the project's reference-free pangenome build.

  \begin{enumerate} % bio2

  \item 1000 Genomes Project Consortium et al. \emph{A map of human genome variation from population-scale sequencing}. \hijournal{Nature}, 467(7319):1061, 2010. PMCID: PMC3042601

  \item 1000 Genomes Project Consortium et al. \emph{An integrated map of genetic variation from 1,092 human genomes}. \hijournal{Nature}, 491(7422):56, 2012. PMCID: PMC3498066

  \item Peter Sudmant, Tobias Rausch, Eugene J Gardner, Bob Handsaker et al. \emph{An integrated map of structural variation in 2,504 human genomes}. \hijournal{Nature}, 526(7571):75, 2015. PMCID: PMC4617611

  \item Human Pangenome Reference Consortium et al. \emph{A Draft Human Pangenome Reference.} \hijournal{bioRxiv} 2022.

  \end{enumerate}


\item Development of a graph-based framework for pangenomic analysis and genome inference

  Variation graphs are sequence graphs with embedded paths (or walks) that describe sequences or genomes of interest. The graphical model provides information about relationship and variation, while the embedded paths provide identity, coordinates, and a mapping between the graphical model and linear reference genomes embedded in it. During my PhD, I have developed efficient algorithms and thrifty data structures that allow these graphs to be used as genomic reference systems at the scale of large eukaryotic genomes, or in the context of topological complexity typical of assembly graphs. Along with numerous collaborators that are part of the VG project, I have shown that variation graphs built from human genomes can improve read mapping performance, and that the best performance occurs when the variation in the graph is a good match for that found in a newly sequenced individual. Our method, the variation graph toolkit VG, provides a reference implementation for the set of tools required to use graphical pangenomic data models in all stages of genome analysis.

My PhD thesis explores applications of VG to a wide array of biological contexts, including standard resequencing, structural variation (SV) detection, ancient DNA, ChIP-Seq analysis, RNA expression profiling, metagenomics, and pangenomics. This diversity of applications is possible due to VG’s ability to use any sequence graph as a reference system, including those produced by resequencing and variant calling, genome alignment, or assembly algorithms. In the context of small variation, I demonstrate that VG can completely remove reference bias for variants in the graph, allowing equal numbers of reads to map to both alleles at heterozygous loci in real samples, regardless of allele length. I show that VG can detect SVs directly in alignment, and that this produces a significant improvement in alignment performance relative the use of a linear reference. Through simulation, I verify that variation aware mapping is more accurate than alignment to a linear reference, and that the overall performance of alignment is not substantially slower than standard sensitive linear genome mappers.

This toolset has provided a common framework around which collaborators have nucleated numerous related explorations. We have established a generic method for the determination of genetic sites that is applicable to any kind of graph structure, which provides a sound basis for genotype and likelihood spaces over sequence data sets. We have developed data structures that provide efficient compression of large numbers of genomes expressed as paths in variation graphs, and are exploring algorithms that enable the use of these haplotype sets in the process of read mapping, variant detection, and genome assembly.

In further work, I am exploring the development of linked data techniques that allow us to directly annotate pieces of a pangenome graph as part of a semantic web database. This should allow us to integrate phenotypic information and the sequence graph model. I aim to provide free software to the research and clinical community. I am strongly committed to working in the open, sharing my results, and providing reproducible software tools to allow others to replicate and generalize my analyses to new domains and applications.

  \begin{enumerate} % bio3

    %  \item Adam M Novak, \textsc{Erik Garrison}, and Benedict Paten. \emph{A graph extension of the positional burrows–wheeler transform and its applications}. \hijournal{Algorithms for Molecular Biology}, 12(1):18, 2017.

  \item Glenn Hickey, David Heller, Jean Monlong, Jonas A. Sibbesen, Jouni Sirén, Jordan Eizenga, Eric T. Dawson, \textsc{Erik Garrison}, Adam M. Novak and Benedict Paten. \emph{Genotyping structural variants in pangenome graphs using the vg toolkit}. \hijournal{Genome Biology}, 21(1), 2020. PMCID: PMC7017486

  \item Jouni Sirén, \textsc{Erik Garrison}, Adam M Novak, Benedict Paten, and Richard Durbin. \emph{Haplotype-aware graph indexes}. \hijournal{Bioinformatics}, 36(2):400, 2019. PMCID: PMC7223266

%  \item Shilpa Garg, Mikko Rautiainen, Adam M Novak, \textsc{Erik Garrison}, Richard Durbin, and Tobias Marschall. \emph{A graph-based approach to diploid genome assembly}. \hijournal{Bioinformatics}, 34(13):i105–i114, 2018. PMCID: PMC5505026

%  \item Benedict Paten, Jordan M Eizenga, Yohei M Rosen, Adam M Novak, \textsc{Erik Garrison}, and Glenn Hickey. \emph{Superbubbles, ultrabubbles, and cacti}. \hijournal{Journal of Computational Biology}, 25(7):649–663, 2018. PMCID: PMC6067107

  \item \textsc{Erik Garrison}, Jouni Sirén, Adam M Novak, Glenn Hickey, Jordan M Eizenga, Eric T Dawson, William Jones, Shilpa Garg, Charles Markello, Michael F Lin, et al. \emph{Variation graph toolkit improves read mapping by representing genetic variation in the reference}. \hijournal{Nature Biotechnology}, 36(9):875–879, 2018. PMCID: PMC6126949

  \item Jouni Sir{\'e}n, Jean Monlong, Xian Chang, Adam Novak, Jordan Eizenga, Charles Markello, Jonas Sibbesen, Glenn Hickey, et al. \emph{Pangenomics enables genotyping of known structural variants in 5202 diverse genomes}. \hijournal{Science}, 374(6574):abg887, 2021. PMCID: PMC34914532

  \end{enumerate}

\item Methods to build and understand lossless, unbiased pangenome graphs

  My work in on graph-based genome analysis demonstrated the feasibility of the approach, leading to a wave of subsequent research on this topic.
  These tools, and new high-quality assemblies implied the need for methods to build pangenome graph reference systems.
  Although several interesting approaches arose that can construct a variation graph (or similar) system, these methods manage to build these graphs only by using a guiding reference system to simplify the problem.
  This perpetuates issues of reference bias that drove me to work on the problem in the first place.
  In response, I have focused my work as an independent PI on establishing and applying methods to build pangenome variation graphs that are completely symmetric and unbiased with respect to all included genomes.
  These methods are not only unbiased, but conceptually simpler for researchers to use than methods based around guide trees or progressive alignment to a reference.
  To develop them, I have engaged with numerous genomics research communities focusing in both human, model, and non-model organisms, with species and clade targets spread across the tree of life.
  %They are rapidly becoming standard methods across numerous
  %I have promulgated and supported the development of tools that allow

  In support of this objective, I guided these works as senior author.
  First, we provided the community with a deep review of all existing methodologies, yielding a technical synthesis that has served as a guide for researchers in the field.
  Then, to work up to the problem of building and manipulating pangenome graphs for research tasks, I led the development of a software library and set of efficient data structures for representing pangenome graphs.
  Finally, in the last year of full independence at UTHSC, I established a modular process to build and normalize pangenome graphs built from any kind of related input genomes, both within the same species or between species.
  This method, the PanGenome Graph Builder (PGGB) is the first approach that breaks down the steps of graph construction into unique modules.
  First, an alignment is made, and this is then converted into a graph.
  Finally, the graph is normalized and can be interrogated by standard, generic methods that work on graphs using standard formats for the representation of assembly and variation graphs.
  I envision this paradigm for whole genome sequence analysis as the foundation for a new kind of genome analysis that responds to current growing needs in bioinformatics.
  Although PGGB is not yet in preprint, we have put forward two components---the ODGI toolkit used to manipulate and understand the graphs, and the seqwish unbiased graph builder.
  I will continue to drive this work forward, working with diverse genomic research communities focused on a variety of model and non-model organisms.
  As the problem of building the graphs closes, I will move my focus to downstream applications of these models that will for the first time let us consider \textit{all} variation in complete genome assemblies.

  \begin{enumerate}

    \item Jordan M. Eizenga, Adam M. Novak, Jonas A. Sibbesen, Simon Heumos, Ali Ghaffaari, Glenn Hickey, Xian Chang, Josiah D Seaman, Robin Rounthwaite, Jana Ebler, Mikko Rautiainen, Shilpa Garg, Benedict Paten, Tobias Marschall, Jouni Sirén, and \textsc{Erik Garrison}. \emph{Pangenome graphs.} \hijournal{Annual review of genomics and human genetics} 21:139-162, 2020. PMCID: PMC7850124

  \item Jordan M. Eizenga, Adam M. Novak, Emily Kobayashi, Flavia Villani, Cecilia Cisar, Simon Heumos, Glenn Hickey, Vincenza Colonna, Benedict Paten, and \textsc{Erik Garrison}. \emph{Efficient dynamic variation graphs.} \hijournal{Bioinformatics} 36(21):5139-5144, 2021. PMCID: PMC8006571

  \item Andrea Guarracino, Simon Heumos, Sven Nahnsen, Pjotr Prins, and \textsc{Erik Garrison}. \emph{ODGI: understanding pangenome graphs.} \hijournal{bioRxiv} 2021.

  \item \textsc{Erik Garrison}, and Andrea Guarracino. \emph{Unbiased pangenome graphs.} \hijournal{bioRxiv} 2022.
  \end{enumerate}

\end{enumerate}

\subsection*{Complete List of Published Work}

\begin{datetbl}
%  Code & \url{https://github.com/ekg} \\
  %Papers & Erik Garrison (as of June 2021)
  \url{https://www.ncbi.nlm.nih.gov/sites/myncbi/erik.garrison.1/collections/60872441/public/}
\end{datetbl}


%------------------------------------------------------------------------------

\begin{comment}
\section{Research Support}

\subsection*{Ongoing Research Support}

\begin{datetbl}
  2020--2023 & U01HG010961 (H. Li and B. Paten) National Human Genome Research Institute, National Institutes of Health. The construction and utility of reference pan-genome graphs.
  We provide systems to support the Human Pangenome Reference Consortium's effort to build unified representations of 300 fully-assembled human genomes.
  Role: coinvestigator
  \\
  2020 & nlnet foundation NGI0 Discovery Fund, Privacy-preserving variation graphs \url{https://nlnet.nl/project/VariationGraph/}.
  Design and implementation of methods to build variation graph models that preserve the privacy of the genomes from which they are built.
  Simultaneously, we are exploring extensions of this approach to mobility data for use in epidemic surveillance.
  Role: principal investigator
\end{datetbl}

% \grantinfo{P30DA044223-01}{ (Saba \& Williams (MPIs))}{07/15/2017–06/30/2022} {Overall NIDA Core ‘Center
%   of Excellence” in Transcriptomics, Systems Genetics and the
%   Addictome} {Genomic and computations support to NIDA research teams
%   using rat models to study addiction. Creation of open access tools
%   for genomic and genetic analysis with a focus on rat research
%   communities.}  {Role:co-investigator}

% \bigskip

% \grantinfo{R01GM123489}{Sen \& Williams, MPIs)}{04/15/2017 – 03/31/2021}
%           {NIH/NIGMS}
%           {A Unified High Performance Web Service for Systems Genetics and Precision Medicine
% This grant funds major software enhancements of GeneNetwork version}
% {Role: Coinvestigator}

\subsection*{Completed Research Support}

\begin{datetbl}
2015--2019 & DT06172015 (D. Haussler) W. M. Keck Foundation. Human Genome Variation Map. We built a comprehensive new representation of human genome variation, the Human Genome Variation Map (HGVM)—a critical new resource to be made available to the scientific and medical communities.  Role: coinvestigator.

%  2018 & {Digital Science Award} Special Blockchain Technology Catalyst Grant with Alexander Garcia Castro (BASF). \$24 K matched by \$60 K for software development from the GeneNetwork project as chief scientific officer % \href{https://www.digital-science.com/blog/news/digital-science-award-special-blockchain-technology-catalyst-grants/}{\ding{234}} \\ % , 20\% effort \\

\end{datetbl}
\end{comment}

\end{document}
